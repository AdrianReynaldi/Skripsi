\documentclass[a4paper,twoside]{article}
\usepackage[T1]{fontenc}
\usepackage[bahasa]{babel}
\usepackage{graphicx}
\usepackage{graphics}
\usepackage{float}
\usepackage[cm]{fullpage}
\pagestyle{myheadings}
\usepackage{etoolbox}
\usepackage{setspace} 
\setlength{\headsep}{30pt}
\usepackage[inner=2cm,outer=2.5cm,top=2.5cm,bottom=2cm]{geometry} %margin
% \pagestyle{empty}

\makeatletter
\renewcommand{\@maketitle} {\begin{center} {\LARGE \textbf{ \textsc{\@title}} \par} \bigskip {\large \textbf{\textsc{\@author}} }\end{center} }
\renewcommand{\thispagestyle}[1]{}
\markright{\textbf{\textsc{AIF401 \textemdash Rencana Kerja Skripsi \textemdash Sem. Ganjil 2013/2014}}}

\onehalfspacing
 
\begin{document}

\title{\@judultopik}
\author{\nama \textendash \@npm} 

%tulis nama dan NPM anda di sini:
\newcommand{\nama}{Lionov}
\newcommand{\@npm}{1997730020}
\newcommand{\@judultopik}{Median Trajectory} % Judul/topik anda
\newcommand{\jumpemb}{1} % Jumlah pembimbing, 1 atau 2
\newcommand{\tanggal}{01/01/1900}
\maketitle

\pagenumbering{arabic}

\section{Deskripsi Skripsi}
{\it Trajectory} adalah jalur dari sebuah obyek yang bergerak. Analisis pada data-data {\it trajectory} merupakan bagian penting dari aplikasi yang di dalamnya terdapat pemrosesan data-data benda bergerak(seperti binatang, moda transportasi, olahraga dll).
Walaupun penelitian mengenai Ada beberapa konsep penting \ldots (to be continued)

\section{Deskripsi Perangkat Lunak}
Perangkat lunak akhir yang akan dibuat memiliki fitur minimal sebagai berikut:
\begin{itemize}
	\item Pengguna dapat membangkitkan data-data {\it trajectory} secara otomatis sesuai dengan aturan yang telah ditentukan.
	\item Pengguna dapat memasukkan secara manual data-data {\it trajectory}, baik melalui suatu {\it GUI (Graphical User Interface)} maupun melalui file teks. 
	\item PL dapat menampilkan data-data yang sudah dimasukkan ataupun yang dibangkitkan secara otomatis melalui {\it GUI (Graphical User Interface)}.
	\item Pengguna dapat memasukkan parameter-parameter yang digunakan oleh algoritma.
	\item PL dapat menghitung dan menampilkan median trajectory secara otomatis, berdasarkan data-data trajectory yang diberikan.
	\item PL dapat secara otomatis melakukan pembangkitan data untuk digunakan pada beberapa ratus tes kasus yang akan diuji. Untuk setiap tes kasus, PL dapat membuat laporan lengkap mengenai tes kasus tersebut.
\end{itemize}

\section{Hal-Hal Yang Akan Dikerjakan}
Hal-hal yang harus dikerjakan untuk menyelesaikan skripsi ini:
\begin{itemize}
	\item Merancang algoritma/langkah-langkah untuk membuat secara otomatis pembangkit data trajectory
	\item Mempelajari algoritma untuk menghitung Frechet Distance
	\item Mempelajari fitur-fitur bahasa Java untuk membuat Graphical User Interface
\end{itemize}

\section{Isi {\it Progress Report 1}}
Isi dari Progress Report 2 yang akan diselesaikan paling lambat pada tanggal 4 Desember 2013 adalah :
\begin{enumerate}
	\item Algoritma/langkah-langkah untuk membuat pembangkit otomatis data trajectory
	\item Hasil eksperimen penggunaan fitur-fitur Graphical User Interface pada bahasa Java
	\item \ldots (to be continued)
\end{enumerate}
Estimasi penyelesaian sampai Progress Report 2 adalah : 20\%

\section{Isi {\it Progress Report 2}}
Isi dari Progress Report 2 yang akan diselesaikan paling lambat pada tanggal 4 Desember 2013 adalah :
\begin{enumerate}
	\item Algoritma dan contoh perhitungan untuk kasus menghitung jarak dengan Frechet Distance
	\item \ldots (to be continued)
\end{enumerate}
Estimasi persentase penyelesaian sampai Progress Report 2 adalah : 45\%
\vspace{1.5cm}

\centering Bandung, \tanggal\\
\vspace{2cm} \nama \\ 
\vspace{1cm}

Menyetujui, \\
\ifdefstring{\jumpemb}{2}{
\vspace{1.5cm}
\begin{centering} Menyetujui,\\ \end{centering} \vspace{0.75cm}
\begin{minipage}[b]{0.45\linewidth}
% \centering Bandung, \makebox[0.5cm]{\hrulefill}/\makebox[0.5cm]{\hrulefill}/2013 \\
\vspace{2cm} Nama: \makebox[3cm]{\hrulefill}\\ Pembimbing Utama
\end{minipage} \hspace{0.5cm}
\begin{minipage}[b]{0.45\linewidth}
% \centering Bandung, \makebox[0.5cm]{\hrulefill}/\makebox[0.5cm]{\hrulefill}/2013\\
\vspace{2cm} Nama: \makebox[3cm]{\hrulefill}\\ Pembimbing Pendamping
\end{minipage}
\vspace{0.5cm}
}{
% \centering Bandung, \makebox[0.5cm]{\hrulefill}/\makebox[0.5cm]{\hrulefill}/2013\\
\vspace{2cm} Nama: \makebox[3cm]{\hrulefill}\\ Pembimbing Tunggal
}
`
\end{document}

