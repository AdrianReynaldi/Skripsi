\documentclass[a4paper,twoside]{article}
\usepackage[T1]{fontenc}
\usepackage[bahasa]{babel}
\usepackage{graphicx}
\usepackage{graphics}
\usepackage{float}
\usepackage[cm]{fullpage}
\pagestyle{myheadings}
\usepackage{etoolbox}
\usepackage{setspace} 
\usepackage{lipsum} 
\setlength{\headsep}{30pt}
\usepackage[inner=2cm,outer=2.5cm,top=2.5cm,bottom=2cm]{geometry} %margin
% \pagestyle{empty}

\makeatletter
\renewcommand{\@maketitle} {\begin{center} {\LARGE \textbf{ \textsc{\@title}} \par} \bigskip {\large \textbf{\textsc{\@author}} }\end{center} }
\renewcommand{\thispagestyle}[1]{}
\markright{\textbf{\textsc{Laporan Perkembangan Pengerjaan Skripsi\textemdash Sem. Ganjil 2015/2016}}}

\onehalfspacing
 
\begin{document}

\title{\@judultopik}
\author{\nama \textendash \@npm} 

%ISILAH DATA DATA BERIKUT INI:
\newcommand{\nama}{Lionov}
\newcommand{\@npm}{1997730020}
\newcommand{\tanggal}{01/01/1900} %Tanggal pembuatan dokumen
\newcommand{\@judultopik}{Simulasi Kerumunan di Museum} % Judul/topik anda
\newcommand{\kodetopik}{TAB0901}
\newcommand{\jumpemb}{1} % Jumlah pembimbing, 1 atau 2
\newcommand{\pembA}{Thomas Anung Basuki}
\newcommand{\pembB}{-}
\newcommand{\semesterPertama}{39 - Ganjil 15/16} % semester pertama kali topik diambil, angka 1 dimulai dari sem Ganjil 96/97
\newcommand{\lamaSkripsi}{1} % Jumlah semester untuk mengerjakan skripsi s.d. dokumen ini dibuat
\newcommand{\kulPertama}{Skripsi 1} % Kuliah dimana topik ini diambil pertama kali
\newcommand{\tipePR}{B} % tipe progress report :
% A : dokumen pendukung untuk pengambilan ke-2 di Skripsi 1
% B : dokumen untuk reviewer pada presentasi dan review Skripsi 1
% C : dokumen pendukung untuk pengambilan ke-2 di Skripsi 2
\maketitle

\pagenumbering{arabic}

\section{Data Skripsi} %TIDAK PERLU MENGUBAH BAGIAN INI !!!
Pembimbing utama/tunggal: {\bf \pembA}\\
Pembimbing pendamping: {\bf \pembB}\\
Kode Topik : {\bf \kodetopik}\\
Topik ini sudah dikerjakan selama : {\bf \lamaSkripsi} semester\\
Pengambilan pertama kali topik ini pada : Semester {\bf \semesterPertama} \\
Pengambilan pertama kali topik ini di kuliah : {\bf \kulPertama} \\
Tipe Laporan : {\bf \tipePR} -
\ifdefstring{\tipePR}{A}{
			Dokumen pendukung untuk {\BF pengambilan ke-2 di Skripsi 1} }
		{
		\ifdefstring{\tipePR}{B} {
				Dokumen untuk reviewer pada presentasi dan {\bf review Skripsi 1}}
			{	Dokumen pendukung untuk {\bf pengambilan ke-2 di Skripsi 2}}
		}

\section{Detail Perkembangan Pengerjaan Skripsi}
Detail bagian pekerjaan skripsi sesuai dengan rencan kerja/laporan perkembangan terkahir :
	\begin{enumerate}
		\item Melakukan survei ke Museum Geologi Bandung untuk mendapatkan denah serta mengetahui perilaku pengunjung museum secara umum (arah perjalanan, kecepatan, lama melihat objek, dll).\\
		{\bf status :} Ada sejak rencana kerja skripsi.\\
		{\bf hasil :} Survei sudah dilakukan sebanyak 3x pada tanggal X, Y dan Z. Pada kunjungan pertama, diperhaitkan denah museum dan dibuat sketsa berdasarkan pengamatan. Pada kunjunga kedua, bertemu dengan bagian humas museum dan berhasil mendapatkan denah serta melakukan wawancara. Hasil wawancara sudah dibuat dan ada softcopy-nya. Pada kunjungan ketiga, secara khusus dilihat perilaku pengunjung. Masih direncakanan 2x kunjungan lagi. Bukti-bukti kunjungan dapat dilihat di lampiran
		
		\item Melakukan analisis pada hasil survei terhadap pergerakan pengunjung di museum dan membuat rancangan denah di komputer yang dilengkapi dengan penghalang dan objek di museum.\\
		{\bf status :} Ada sejak rencana kerja skripsi.\\
		{\bf hasil :}

		\item Melakukan studi literatur mengenai sifat kolektif suatu kerumunan, teknik {\it social force model} dan teknik {\it flow tiles}\\
		{\bf status :} Ada sejak rencana kerja skripsi.\\
		{\bf hasil :}

		\item Mempelajari bahasa pemrograman C++ dan cara menggunakan framework OpenSteer\\
		{\bf status :} Ada sejak rencana kerja skripsi.\\
		{\bf hasil :}

		\item Merancang pergerakan kerumunan di dalam museum menggunakan teknik {\it social force model} dan {\it flow tiles} serta menggunakan teknik lainnya seperti konsep pathway dan waypoints. Selain itu, dirancang pula adanya waktu tunggu (pada saat pengunjung melihat objek di museum) dan cara pembuatan jalur bagi setiap individu pengunjung\\
		{\bf status :} Ada sejak rencana kerja skripsi.\\
		{\bf hasil :}

		\item Melakukan analisa dan merancang struktur data yang cocok untuk menyimpan penghalang (obstacle)\\
		{\bf status :} dihapuskan/tidak dikerjakan \\
		{\bf hasil :} berdasarkan analisis singkat, tidak dilakukan analisis lebih jauh karena tidak diperlukan struktur data baru, karena sudah disediakan oleh OpenSteer versi terbaru

		\item Mengimplementasikan keseluruhan algoritma dan struktur data yang dirancang, dengan menggunakan framework OpenSteer \\
		{\bf status :} Ada sejak rencana kerja skripsi.\\
		{\bf hasil :}

		\item Melakukan pengujian (dan eksperimen) yang melibatkan responde untuk menilai hasil simulasi secara kualitatif\\
		{\bf status :} Ada sejak rencana kerja skripsi.\\
		{\bf hasil :}

		\item Menulis dokumen skripsi\\
		{\bf status :} Ada sejak rencana kerja skripsi.\\
		{\bf hasil :} \lipsum[1]
		
		\item Mempelajari cara menggunakan fitur manipulasi obstacle yang disediakan oleh framework Opensteer versi terbaru\\
		{\bf status :} baru ditambahkan pada semester ini\\
		{\bf hasil :} baru direncanakan karena framework Opensteer versi paling akhir baru selesai diinstall dan dilihat-lihat bagian contoh-contoh simulasinya
		

	\end{enumerate}

\section{Pencapaian Rencana Kerja}
Persentase penyelesaian skripsi sampai dengan dokumen ini dibuat dapat dilihat pada tabel berikut :

\begin{center}
  \begin{tabular}{ | c | c | c | c | l | c |}
    \hline
    1*  & 2*(\%) & 3*(\%) & 4*(\%) &5* &6*(\%)\\ \hline \hline
    1   & 5  & 5  &  &  & 3 \\ \hline
    2   & 5 & 5  &   &  & 3 \\ \hline
    3   & 10  & 7  & 3 & {\footnotesize sebagian kecil teknik {\it flow tiles} di S2} & 5 \\ \hline
    4   & 15  & 10  &  5 & {\footnotesize teknik lanjut OOP di C++ di S2} & 12 \\ \hline
    5   & 20  & 5  & 15 & {\footnotesize perancangan awal SFM, pathway dan waypoint di S1} & 5 \\ \hline
    6   & 0 &   & 0  &  {\footnotesize dihapus karena tidak diperlukan} & \\\hline
    7   & 20  & 5  & 15 &  {\footnotesize implementasi denah dan rancangan awal SFM di S1} & 3 \\ \hline
    8   & 5  &   &  5  &  &\\ \hline
    9   & 15  & 3  & 12  & {\footnotesize sebagian bab 1 dan 2, serta bagian awal analisis di S1} & 5\\ \hline
    10   & 5  &   & 5  & {\footnotesize tambahan baru} &\\ \hline
    Total  & 100  & 40  & 60 &  & 36\\ \hline
                          \end{tabular}
\end{center}

Keterangan (*)\\
1 : Bagian pengerjaan Skripsi (nomor disesuaikan dengan detail pengerjaan di bagian 5)\\
2 : Persentase total \\
3 : Persentase yang akan diselesaikan di Skripsi 1 \\
4 : Persentase yang akan diselesaikan di Skripsi 2 \\
5 : Penjelasan singkat apa yang dilakukan di S1 (Skripsi 1) atau S2 (skripsi 2)\\
6 : Persentase yang sidah diselesaikan sampai saat ini 

\section{Kendala yang dihadapi}
%TULISKAN BAGIAN INI JIKA DOKUMEN ANDA TIPE A ATAU C
Kendala - kendala yang dihadapi selama mengerjakan skripsi :
\begin{itemize}
	\item Terlalu banyak melakukan prokratinasi
	\item Terlalu banyak godaan berupa hiburan (game, film, dll)
	\item Skripsi diambil bersamaan dengan kuliah ASD karena selama 5 semester pertama kuliah tersebut sangat dihindari dan tidak diambil, dan selama 4 semester terakhir kuliah tersebut selalu mendapat nilai E
	\item Mengalami kesulitan pada saat sudah mulai membuat program komputer karena selama ini selalu dibantu teman
\end{itemize}

\vspace{1cm}
\centering Bandung, \tanggal\\
\vspace{2cm} \nama \\ 
\vspace{1cm}

Menyetujui, \\
\ifdefstring{\jumpemb}{2}{
\vspace{1.5cm}
\begin{centering} Menyetujui,\\ \end{centering} \vspace{0.75cm}
\begin{minipage}[b]{0.45\linewidth}
% \centering Bandung, \makebox[0.5cm]{\hrulefill}/\makebox[0.5cm]{\hrulefill}/2013 \\
\vspace{2cm} Nama: \pembA \\ Pembimbing Utama
\end{minipage} \hspace{0.5cm}
\begin{minipage}[b]{0.45\linewidth}
% \centering Bandung, \makebox[0.5cm]{\hrulefill}/\makebox[0.5cm]{\hrulefill}/2013\\
\vspace{2cm} Nama: \pemB \\ Pembimbing Pendamping
\end{minipage}
\vspace{0.5cm}
}{
% \centering Bandung, \makebox[0.5cm]{\hrulefill}/\makebox[0.5cm]{\hrulefill}/2013\\
\vspace{2cm} Nama: \pembA \\ Pembimbing Tunggal
}

\end{document}

