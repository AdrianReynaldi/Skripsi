\documentclass[a4paper,twoside]{article}
\usepackage[T1]{fontenc}
\usepackage[bahasa]{babel}
\usepackage{graphicx}
\usepackage{graphics}
\usepackage{float}
\usepackage[cm]{fullpage}
\pagestyle{myheadings}
\usepackage{etoolbox}
\usepackage{setspace} 
\usepackage{lipsum} 
\setlength{\headsep}{30pt}
\usepackage[inner=2cm,outer=2.5cm,top=2.5cm,bottom=2cm]{geometry} %margin
% \pagestyle{empty}

\makeatletter
\renewcommand{\@maketitle} {\begin{center} {\LARGE \textbf{ \textsc{\@title}} \par} \bigskip {\large \textbf{\textsc{\@author}} }\end{center} }
\renewcommand{\thispagestyle}[1]{}
\markright{\textbf{\textsc{Laporan Perkembangan Pengerjaan Skripsi\textemdash Sem. Genap 2016/2017}}}

\onehalfspacing
 
\begin{document}

\title{\@judultopik}
\author{\nama \textendash \@npm} 

%ISILAH DATA DATA BERIKUT INI:
\newcommand{\nama}{Adrian Reynaldi}
\newcommand{\@npm}{2013730058}
\newcommand{\tanggal}{08/05/2017} %Tanggal pembuatan dokumen
\newcommand{\@judultopik}{Aplikasi Pembangkit Jadwal Dosen} % Judul/topik anda
\newcommand{\kodetopik}{PAN4203}
\newcommand{\jumpemb}{1} % Jumlah pembimbing, 1 atau 2
\newcommand{\pembA}{Pascal Alfadian}
\newcommand{\pembB}{-}
\newcommand{\semesterPertama}{42 - Genap 16/17} % semester pertama kali topik diambil, angka 1 dimulai dari sem Ganjil 96/97
\newcommand{\lamaSkripsi}{1} % Jumlah semester untuk mengerjakan skripsi s.d. dokumen ini dibuat
\newcommand{\kulPertama}{Skripsi 1} % Kuliah dimana topik ini diambil pertama kali
\newcommand{\tipePR}{B} % tipe progress report :
% A : dokumen pendukung untuk pengambilan ke-2 di Skripsi 1
% B : dokumen untuk reviewer pada presentasi dan review Skripsi 1
% C : dokumen pendukung untuk pengambilan ke-2 di Skripsi 2
\maketitle

\pagenumbering{arabic}

\section{Data Skripsi} %TIDAK PERLU MENGUBAH BAGIAN INI !!!
Pembimbing utama/tunggal: {\bf \pembA}\\
Pembimbing pendamping: {\bf \pembB}\\
Kode Topik : {\bf \kodetopik}\\
Topik ini sudah dikerjakan selama : {\bf \lamaSkripsi} semester\\
Pengambilan pertama kali topik ini pada : Semester {\bf \semesterPertama} \\
Pengambilan pertama kali topik ini di kuliah : {\bf \kulPertama} \\
Tipe Laporan : {\bf \tipePR} -
\ifdefstring{\tipePR}{A}{
			Dokumen pendukung untuk {\BF pengambilan ke-2 di Skripsi 1} }
		{
		\ifdefstring{\tipePR}{B} {
				Dokumen untuk reviewer pada presentasi dan {\bf review Skripsi 1}}
			{	Dokumen pendukung untuk {\bf pengambilan ke-2 di Skripsi 2}}
		}

\section{Detail Perkembangan Pengerjaan Skripsi}
Detail bagian pekerjaan skripsi sesuai dengan rencan kerja/laporan perkembangan terkahir :

	\begin{enumerate}
		\item Melakukan studi literatur mengenai PHP dan \textit{Codeigniter}, \textit{Zurb Foundation}, \textit{Goggle OAuth} \textit{PHPExcel}\\
		      \textbf{status} : Ada sejak rencana kerja skripsi.\\
		      \textbf{hasil} : Studi literatur sudah dilakukan dengan membaca-baca dokumentasinya dan langsung dicoba untuk diimplementasikan pada modul-modul.
		
		\item Membuat flowchart pembangkitan jadwal dosen \\
		    \textbf{status} : Ada sejak rencana kerja skripsi.\\
		    \textbf{hasil} : 
		\item Membuat flowchart perangkat lunak. \\
		    \textbf{status} : Ada sejak rencana kerja skripsi.\\
		    \textbf{hasil} :
		\item Membuat rancangan diagram kelas. \\
		    \textbf{status} : Ada sejak rencana kerja skripsi.\\
		    \textbf{hasil} : rancangan kelas sudah jadi, namun belum didokumentasikan.
		\item Membuat rancangan basis data.
		    \textbf{status} : Ada sejak rencana kerja skripsi.\\
		    \textbf{hasil} : rancangan basis data sudah jadi, namun belum didokumentasikan.
		\item Mengimplementasikan keseluruhan rancangan dan modul-modulnya \\
		    \textbf{status} : Ada sejak rencana kerja skripsi.\\
		    \textbf{hasil} : semua fitur-fitur utama modul sudah selesai dikerjakan dan sudah dapat dipakai oleh pengguna. Implementasi ini diselesaikan lebih cepat dari rencana pada rencana kerja skripsi karena modul perlu diuji coba oleh dosen-dosen dan masukan dari mereka akan digunakan di skripsi 2. Bagian ini belum selesai karena adanya kemungkinan penambahan fitur dari masukan-masukan pengguna.
		\item Menguji perangkat lunak \\
		    \textbf{status} : Ada sejak rencana kerja skripsi.\\
		    \textbf{hasil} : 
		\item Menulis dokumen skripsi \\
		    \textbf{status} : Ada sejak rencana kerja skripsi.\\
		    \textbf{hasil} : Penulisan batasan masalah pada Bab 1 masih belum selesai. Penulisan dasar teori mengenai Foundation, Codeigniter dan PHPExcel masih perlu dilengkapi.
	\end{enumerate}


\section{Pencapaian Rencana Kerja}
Persentase penyelesaian skripsi sampai dengan dokumen ini dibuat dapat dilihat pada tabel berikut :

\begin{center}
  \begin{tabular}{ | c | c | c | c | l | c |}
    \hline
    1*  & 2*(\%) & 3*(\%) & 4*(\%) &5* &6* \\ \hline \hline
    1   &  10 	&  10 &   &  & 10 \\ \hline
    2   &  5 	&  5 &   &  & 0 \\ \hline
	3   &  5 	&  5 &   & {\footnotesize }   & 0 \\ \hline
	4   &  10 	&   & 10  & {\footnotesize }  & 5 \\ \hline
	5   &  10 	&   & 10  & {\footnotesize }  & 5 \\ \hline
	6   &  35  	& 15 & 20  & {\footnotesize } & 25 \\ \hline
	7   &  10 	&   &  10 & {\footnotesize }  & 3 \\ \hline
	8   &  15 	& 7 & 8& {\footnotesize Sebagian bab 1, 2, dan 3 dikerjakan di skripsi 1 }  & 3 \\ \hline
    Total  & 100  & 42  & 58 &  & 51 \\ \hline
                          \end{tabular}
\end{center}


Keterangan (*)\\
1 : Bagian pengerjaan Skripsi (nomor disesuaikan dengan detail pengerjaan di bagian 5)\\
2 : Persentase total \\
3 : Persentase yang akan diselesaikan di Skripsi 1 \\
4 : Persentase yang akan diselesaikan di Skripsi 2 \\
5 : Penjelasan singkat apa yang dilakukan di S1 (Skripsi 1) atau S2 (skripsi 2)
6 : Persentase yang sudah diselesaikan sampai saat ini \\

\vspace{1cm}
\centering Bandung, \tanggal\\
\vspace{2cm} \nama \\ 
\vspace{1cm}

Menyetujui, \\
\ifdefstring{\jumpemb}{2}{
\vspace{1.5cm}
\begin{centering} Menyetujui,\\ \end{centering} \vspace{0.75cm}
\begin{minipage}[b]{0.45\linewidth}
% \centering Bandung, \makebox[0.5cm]{\hrulefill}/\makebox[0.5cm]{\hrulefill}/2013 \\
\vspace{2cm} Nama: \pembA \\ Pembimbing Utama
\end{minipage} \hspace{0.5cm}
\begin{minipage}[b]{0.45\linewidth}
% \centering Bandung, \makebox[0.5cm]{\hrulefill}/\makebox[0.5cm]{\hrulefill}/2013\\
\vspace{2cm} Nama: \pemB \\ Pembimbing Pendamping
\end{minipage}
\vspace{0.5cm}
}{
% \centering Bandung, \makebox[0.5cm]{\hrulefill}/\makebox[0.5cm]{\hrulefill}/2013\\
\vspace{2cm} Nama: \pembA \\ Pembimbing Tunggal
}

\end{document}

