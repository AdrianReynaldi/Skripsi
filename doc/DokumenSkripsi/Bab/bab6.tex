\chapter{Kesimpulan dan Saran}
\section{Kesimpulan}
\paragraph{} Penelitian yang dilakukan dalam pengembangan Aplikasi Pembangkit Jadwal Dosen berhasil memenuhi harapan dalam mencatat dan menampilkan jadwal dosen. Kesimpulan yang dapat diambil dari hasil penelitian adalah sebagai berikut:
\begin{enumerate}
	\item Aplikasi memenuhi tujuan dalam mengotentikasi pengguna yang mengakses BlueTape. Pengguna dosen dapat mengakses modul Entri Jadwal Dosen dan Lihat Jadwal Dosen sedangkan pengguna mahasiswa hanya dapat mengakses modul Lihat Jadwal Dosen.
	\item Aplikasi memenuhi tujuan menyediakan cara bagi dosen untuk memasukan jadwalnya ke dalam BlueTape dengan mengimplementasikan modul Entri Jadwal Dosen yang berisi menu untuk menambah jadwal. Selain itu, aplikasi ini juga memenuhi tujuan agar jadwal dosen dapat ditampilkan di BlueTape dengan mengimplementasikan modul Lihat Jadwal Dosen yang menampilkan setiap jadwal dosen dalam bentuk tabel-tabel.
	\item Aplikasi memenuhi tujuan untuk mengekspor jadwal yang disimpan di basis data menjadi tipe file xls.
\end{enumerate}

\section{Saran}
\paragraph{}Berdasarkan hasil kesimpulan di atas, maka berikut merupakan saran-saran yang dapat diberikan untuk pengembangan selanjutnya:
\begin{enumerate}
	\item Ditemukan bahwa format email untuk mahasiswa angkatan 2017 memiliki format yang berbeda dengan mahasiswa-mahasiswa angkatan sebelumnya. Mahasiswa angkatan 2017 memiliki format xxxx73yyyy@student.unpar.ac.id xxxx merupakan angkatan dan yyyy merupakan nomor pokoknya. Sedangkan sebelumnya email mahasiswa memilki format 73xxyyyy@student.unpar.ac.id dengan xx adalah angkatan dan yyyy nomor pokoknya. Perlu dikaji ulang cara otentikasi dan pengelompokan pengguna mahasiswa pada penerapan Google OAuth.
	\item Mengurutkan tab-tab jadwal dosen berdasarkan alphabet atau urutan lainnya agar nama dosen mudah untuk dicari.
	\item Mengganti library PHPExcel dengan versi terbarunya yaitu PHPOffice untuk mendukung pembuatan file bertipe .xlsx agar ukuran file lebih kecil.
\end{enumerate}
