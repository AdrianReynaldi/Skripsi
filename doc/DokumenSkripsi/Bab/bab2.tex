%versi 2 (8-10-2016)
\chapter{Landasan Teori}
\label{chap:teori}

\section{Codeigniter}
\label{sec:codeigniter} 
 
\paragraph{} Codeigniter adalah \textit{framework} pengembangan aplikasi untuk orang-orang yang membangun situs web menggunakan PHP. Tujuannya adalah untuk memungkinkan Anda mengembangkan proyek lebih cepat, daripada bila Anda menulis kode dari awal, dengan menyediakan banyak kumpulan \textit{library} untuk tugas-tugas yang sering dibutuhkan dan juga menyediakan tampilan sederhana serta struktur logika untuk mengakses \textit{library-library} tersebut. Codeigniter memungkinkan Anda untuk fokus secara kretif pada proyek Anda dengan cara meminimalkan jumlah kode yang dibutuhkan untuk setiap tugas yang diberikan.
\\
\\
Codeigniter dirancang untuk memenuhi kebutuhan orang-orang yang membutuhkan :
\begin{itemize}
		\item \textit{Framework} dengan tapak keberadaan yang kecil
		\item performa yang baik
		\item kompabilitas akun \textit{hosting} yang luas yang dapat berjalan di berbagai versi dan konfigurasi PHP
		\item Framework yang hampir tidak membutuhkan konfigurasi
		\item \textit{Framework} yang tidak membutuhkan \textit{command line}
		\item Framework yang tidak mengikuti aturan pengkodean yang ketat
		\item membutuhkan solusi yang sederhana
		\item dokumentasi yang menyeluruh
	\end{itemize}
