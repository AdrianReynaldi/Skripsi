%versi 3 (18-12-2016)
\chapter{Kode Program Implementasi Modul Entri Jadwal Dosen}
\label{Implementasi Modul Entri Jadwal Dosen}

%terdapat 2 cara untuk memasukkan kode program
% 1. menggunakan perintah \lstinputlisting (kode program ditempatkan di folder yang sama dengan file ini)
% 2. menggunakan environment lstlisting (kode program dituliskan di dalam file ini)
% Perhatikan contoh yang diberikan!!
%
% untuk keduanya, ada parameter yang harus diisi:
% - language: bahasa dari kode program (pilihan: Java, C, C++, PHP, Matlab, C#, HTML, R, Python, SQL, dll)
% - caption: nama file dari kode program yang akan ditampilkan di dokumen akhir
%
% Perhatian: Abaikan warning tentang textasteriskcentered!!
%

Kode Program untuk \textit{controller} modul Entri Jadwal Dosen
\lstinputlisting[language=php, caption=EntriJadwalDosen.php]{./Lampiran/EntriJadwalDosen.php} 

Kode Program untuk \textit{view} modul Entri Jadwal Dosen
\lstinputlisting[language=php, caption=main.php]{./Lampiran/main.php} 
