%_____________________________________________________________________________
%=============================================================================
% data.tex v7 (30-11-2015) \ldots dibuat oleh Lionov - Informatika FTIS UNPAR
%
% Perubahan pada versi 7 (30-11-2015)
%	- Perubahan nomor bagian karena penyisipan Bagian V
%	- Penambahan Bagian V : pengaturan kemunculan daftar gambar dan/atau tabel 
%	  dipindahkan dari main.tex
%	- Penambahan Bagian XV : tempat untuk menambahkan perintah yang dibuat 
%	  sendiri, dipindahkan dari main.tex
%	- Penambahan Bagian 0 : perintah bagi yang caption daftar isi tidak bisa
%	  ke bagian tengah
%
% Perubahan pada versi sebelumnya dapat dilihat di bagian akhir file ini
%_____________________________________________________________________________
%=============================================================================

%=============================================================================
% 								PETUNJUK
%=============================================================================
% Ini adalah file data (data.tex)
% Masukkan ke dalam file ini, data-data yang diperlukan oleh template ini
% Cara memasukkan data dijelaskan di setiap bagian
% Data yang WAJIB dan HARUS diisi dengan baik dan benar adalah SELURUHNYA !!
% Hilangkan tanda << dan >> jika anda menemukannya
%=============================================================================

%_____________________________________________________________________________
%=============================================================================
% 								BAGIAN 0
%=============================================================================
% PERHATIAN!! PERHATIAN!! Bagian ini hanya ada untuk sementara saja
% Jika "DAFTAR ISI" tidak bisa berada di bagian tengah halaman, isi dengan XXX
% jika sudah benar posisinya, biarkan kosong (i.e. \daftarIsiError{ })
%=============================================================================
\daftarIsiError{ }
%=============================================================================

%_____________________________________________________________________________
%=============================================================================
% 								BAGIAN I
%=============================================================================
% Tambahkan package2 lain yang anda butuhkan di sini
%=============================================================================
\usepackage{booktabs} 
\usepackage[table]{xcolor}
\usepackage{longtable}
\usepackage{amsmath}
\usepackage{todo}
%=============================================================================

%_____________________________________________________________________________
%=============================================================================
% 								BAGIAN II
%=============================================================================
% Mode dokumen: menetukan halaman depan dari dokumen, apakah harus mengandung 
% prakata/pernyataan/abstrak dll (termasuk daftar gambar/tabel/isi) ?
% - kosong : tidak ada halaman depan sama sekali (untuk dokumen yang 
%            dipergunakan pada proses bimbingan)
% - cover : cover saja tanpa daftar isi, gambar dan tabel
% - sidang : cover, daftar isi, gambar, tabel (IT: UTS-UAS Seminar 
%			 dan UTS TA)
% - sidang_akhir : mode sidang + abstrak + abstract
% - final : seluruh halaman awal dokumen (untuk cetak final)
% Jika tidak ingin mencetak daftar tabel/gambar (misalkan karena tidak ada 
% isinya), edit manual di baris 439 dan 440 pada file main.tex
%=============================================================================
% \mode{kosong}
% \mode{cover}
% \mode{sidang}
%\mode{sidang_akhir}
\mode{final} 
%=============================================================================

%_____________________________________________________________________________
%=============================================================================
% 								BAGIAN III
%=============================================================================
% Line numbering: penomoran setiap baris, otomatis di-reset setiap berganti
% halaman
% - yes: setiap baris diberi nomor
% - no : baris tidak diberi nomor, otomatis untuk mode final
%=============================================================================
\linenumber{yes}
%=============================================================================

%_____________________________________________________________________________
%=============================================================================
% 								BAGIAN IV
%=============================================================================
% Linespacing: jarak antara baris 
% - single: opsi yang disediakan untuk bimbingan, jika pembimbing tidak
%            keberatan (untuk menghemat kertas)
% - onehalf: default dan wajib (dan otomatis) jika ingin mencetak dokumen
%            final/untuk sidang.
% - double : jarak yang lebih lebar lagi, jika pembimbing berniat memberi 
%            catatan yg banyak di antara baris (dianjurkan untuk bimbingan)
%=============================================================================
\linespacing{single}
%\linespacing{onehalf}
%\linespacing{double}
%=============================================================================

%_____________________________________________________________________________
%=============================================================================
% 								BAGIAN V
%=============================================================================
% Tidak semua skripsi memuat gambar dan/atau tabel. Untuk skripsi yang seperti
% itu, tidak diperlukan Daftar Gambar dan Daftar Tabel. Sayangnya hal ini 
% sulit dilakukan secara manual karena membutuhkan kedisiplinan pengguna 
% template.  
% Jika tidak akan menampilkan Daftar Gambar/Tabel, isi dengan NO. Jika ingin
% menampilkan, kosongkan parameter (i.e. \gambar{ }, \tabel{ })
%=============================================================================
\gambar{ }
\tabel{ }
%=============================================================================

%_____________________________________________________________________________
%=============================================================================
% 								BAGIAN VI
%=============================================================================
% Bab yang akan dicetak: isi dengan angka 1,2,3 s.d 9, sehingga bisa digunakan
% untuk mencetak hanya 1 atau beberapa bab saja
% Jika lebih dari 1 bab, pisahkan dengan ',', bab akan dicetak terurut sesuai 
% urutan bab (e.g. \bab{1,2,3}).
% Untuk mencetak seluruh bab, kosongkan parameter (i.e. \bab{ })  
% Catatan: Jika ingin menambahkan bab ke-10 dan seterusnya, harus dilakukan 
% secara manual
%=============================================================================
\bab{ }
%=============================================================================

%_____________________________________________________________________________
%=============================================================================
% 								BAGIAN VII
%=============================================================================
% Lampiran yang akan dicetak: isi dengan huruf A,B,C s.d I, sehingga bisa 
% digunakan untuk mencetak hanya 1 atau beberapa lampiran saja
% Jika lebih dari 1 lampiran, pisahkan dengan ',', lampiran akan dicetak 
% terurut sesuai urutan lampiran (e.g. \bab{A,B,C}).
% Jika tidak ingin mencetak lampiran apapun, isi dengan -1 (i.e. \lampiran{-1})
% Untuk mencetak seluruh mapiran, kosongkan parameter (i.e. \lampiran{ })  
% Catatan: Jika ingin menambahkan lampiran ke-J dan seterusnya, harus 
% dilakukan secara manual
%=============================================================================
\lampiran{ }
%=============================================================================

%_____________________________________________________________________________
%=============================================================================
% 								BAGIAN VIII
%=============================================================================
% Data diri dan skripsi/tugas akhir
% - namanpm: Nama dan NPM anda, penggunaan huruf besar untuk nama harus benar
%			 dan gunakan 10 digit npm UNPAR, PASTIKAN BAHWA BENAR !!!
%			 (e.g. \namanpm{Jane Doe}{1992710001}
% - judul : Dalam bahasa Indonesia, perhatikan penggunaan huruf besar, judul
%			tidak menggunakan huruf besar seluruhnya !!! 
% - tanggal : isi dengan {tangga}{bulan}{tahun} dalam angka numerik, jangan 
%			  menuliskan kata (e.g. AGUSTUS) dalam isian bulan
%			  Tanggal ini adalah tanggal dimana anda akan melaksanakan sidang 
%			  ujian akhir skripsi/tugas akhir
% - pembimbing: isi dengan pembimbing anda, lihat daftar dosen di file dosen.tex
%				jika pembimbing hanya 1, kosongkan parameter kedua 
%				(e.g. \pembimbing{\JND}{  } ) , \JND adalah kode dosen
% - penguji : isi dengan para penguji anda, lihat daftar dosen di file dosen.tex
%				(e.g. \penguji{\JHD}{\JCD} ) , \JND dan \JCD adalah kode dosen
% !!Lihat singkatan pembimbing dan penguji anda di file dosen.tex
%=============================================================================
\namanpm{<<Nama Lengkap>>}{<<10 digit NPM UNPAR>>}	%hilangkan tanda << & >>
\tanggal{<<tanggal>>}{<<bulan>>}{<<tahun>>}			%hilangkan tanda << & >>
\pembimbing{<<pembimbing utama/1>>}{<<pembimbing pendamping/2>>} %hilangkan tanda << & >>    
\penguji{<<penguji 1>>}{<<penguji 2>>} 				%hilangkan tanda << & >>
%=============================================================================

%_____________________________________________________________________________
%=============================================================================
% 								BAGIAN IX
%=============================================================================
% Judul dan title : judul bhs indonesia dan inggris
% - judulINA: judul dalam bahasa indonesia
% - judulENG: title in english
% PERHATIAN: - langsung mulai setelah '{' awal, jangan mulai menulis di baris 
%			   bawahnya
%			 - Gunakan \texorpdfstring{\\}{} untuk pindah ke baris baru
%			 - Judul TIDAK ditulis dengan menggunakan huruf besar seluruhnya !!
%			 - Gunakan perintah \texorpdfstring{\\}{} untuk baris baru
%=============================================================================
\judulINA{<<Judul Bahasa Indonesia>>}
\judulENG{<<Judul Bahasa Inggris>>}
%_____________________________________________________________________________
%=============================================================================
% 								BAGIAN X
%=============================================================================
% Abstrak dan abstract : abstrak bhs indonesia dan inggris
% - abstrakINA: abstrak bahasa indonesia
% - abstrakENG: abstract in english
% PERHATIAN: langsung mulai setelah '{' awal, jangan mulai menulis di baris 
%			 bawahnya
%=============================================================================
\abstrakINA{<<Tuliskan abstrak anda di sini, dalam bahasa Indonesia>> \lipsum[5]}
\abstrakENG{<<Tuliskan abstrak anda di sini, dalam bahasa Inggris>> \lipsum[5]} 
%=============================================================================

%_____________________________________________________________________________
%=============================================================================
% 								BAGIAN XI
%=============================================================================
% Kata-kata kunci dan keywords : diletakkan di bawah abstrak (ina dan eng)
% - kunciINA: kata-kata kunci dalam bahasa indonesia
% - kunciENG: keywords in english
%=============================================================================
\kunciINA{<<Tuliskan di sini kata-kata kunci yang anda gunakan, dalam bahasa Indonesia>>}
\kunciENG{<<Tuliskan di sini kata-kata kunci yang anda gunakan, dalam bahasa Inggris>>}
%=============================================================================

%_____________________________________________________________________________
%=============================================================================
% 								BAGIAN XII
%=============================================================================
% Persembahan : kepada siapa anda mempersembahkan skripsi ini ...
%=============================================================================
\untuk{<<kepada siapa anda mempersembahkan skripsi ini\ldots?>>}
%=============================================================================

%_____________________________________________________________________________
%=============================================================================
% 								BAGIAN XIII
%=============================================================================
% Kata Pengantar: tempat anda menuliskan kata pengantar dan ucapan terima 
% kasih kepada yang telah membantu anda bla bla bla ....  
%=============================================================================
\prakata{\lipsum[3]}
%=============================================================================

%_____________________________________________________________________________
%=============================================================================
% 								BAGIAN XIV
%=============================================================================
% Tambahkan hyphen (pemenggalan kata) yang anda butuhkan di sini 
%=============================================================================
\hyphenation{ma-te-ma-ti-ka}
\hyphenation{fi-si-ka}
\hyphenation{tek-nik}
\hyphenation{in-for-ma-ti-ka}
%=============================================================================

%_____________________________________________________________________________
%=============================================================================
% 								BAGIAN XV
%=============================================================================
% Tambahkan perintah yang anda buat sendiri di sini 
%=============================================================================
\newcommand{\vtemplateauthor}{lionov}
%=============================================================================

%=============================================================================
% Perubahan pada versi 6 (13-04-2015)
% 	- Perubahan untuk data-data ``template" menjadi lebih generik dan 
%	  menggunakan tanda << dan >>
% Perubahan pada versi 5 (10-11-2013)
% 	- Perbaikan pada memasukkan bab : tidak perlu menuliskan apapun untuk 
%	  memasukkan seluruh bab (bagian V)
% 	- Perbaikan pada memasukkan lampiran : tidak perlu menuliskan apapun untuk
%	  memasukkan seluruh lampiran atau -1 jika tidak memasukkan apapun
% Perubahan pada versi 4 (21-10-2012)
%	- Data dosen dipindah ke dosen.tex agar jika ada perubahan/update data 
%	  dosen, mahasiswa tidak perlu mengubah data.tex
%	- Perubahan pada keterangan dosen	
% Perubahan pada versi 3 (06-08-2012)
% 	- Perubahan pada beberapa keterangan 
% Perubahan pada versi 2 (09-07-2012):
% 	- Menambahkan data judul dalam bahasa inggris
% 	- Membuat bagian khusus untuk judul (bagian VIII)
% 	- Perbaikan pada gelar dosen
% Versi 1 (08-11-2011)
%=============================================================================